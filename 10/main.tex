\documentclass[12pt]{amsart}
\usepackage{amsmath}
\usepackage{amsthm}
\usepackage{amsfonts}
\usepackage{amssymb}
\usepackage{ebproof}
\usepackage[margin=1in]{geometry}
\usepackage{hyperref}
\hypersetup{
    colorlinks=true,
    linkcolor=blue
}

\theoremstyle{definition}
\newtheorem{theorem}{Theorem}[section]
\newtheorem{lemma}[theorem]{Lemma}
\newtheorem{definition}[theorem]{Definition}
\newtheorem{corollary}[theorem]{Corollary}
\newtheorem{proposition}[theorem]{Proposition}
\newtheorem{conjecture}[theorem]{Conjecture}
\newtheorem{remark}[theorem]{Remark}
\newtheorem{example}[theorem]{Example}
\newtheorem{problem}[theorem]{Problem}
\newtheorem{notation}[theorem]{Notation}
\newtheorem{question}[theorem]{Question}
\newtheorem{caution}[theorem]{Caution}

\begin{document}

\title{Quiz}

\maketitle

Decide whether the following statements are always, sometimes or never true. 
Explain your reasoning. 

\begin{itemize}

\item There is a unique function $X \to \lbrace 0 \rbrace$. 

\item Let $\operatorname{Func}(X,Y)$ be the set of functions with domain $X$ 
	and codomain $Y$. If $Y \cong Z$, then $\operatorname{Func}(X,Y) 
	\cong \operatorname{Func}(X,Z)$. 

\item Let $G \subseteq X \times Y$. There exists a function $f: X \to Y$ such 
	that $\Gamma_f = G$. 

\end{itemize}

\end{document}
